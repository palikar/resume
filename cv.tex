\documentclass[letterpaper,10pt]{article}

\usepackage{latexsym}
\usepackage[empty]{fullpage}
\usepackage{titlesec}
\usepackage{marvosym}
\usepackage[usenames,dvipsnames]{color}
\usepackage{verbatim}
\usepackage{enumitem}
\usepackage[pdftex, hidelinks]{hyperref}
\usepackage{fancyhdr}
\usepackage[export]{adjustbox}
\usepackage{graphicx}
\usepackage{tikz}
\usepackage{wrapfig}
\usepackage[charter]{mathdesign}
\usepackage{numprint}
\pdfpkresolution=450
\pagestyle{fancy}
\fancyhf{}
\fancyfoot{}
\npthousandsep{\,}

\renewcommand{\headrulewidth}{0pt}
\renewcommand{\footrulewidth}{0pt}
\addtolength{\oddsidemargin}{-0.50in}
\addtolength{\evensidemargin}{-0.50in}
\addtolength{\textwidth}{1in}
\addtolength{\topmargin}{-.5in}
\addtolength{\textheight}{1.0in}
\urlstyle{same}
\raggedbottom
\raggedright
\setlength{\tabcolsep}{0in}
\newcommand{\roundpic}[4][]{
  \tikz\node [circle, minimum width = #2,
  path picture = {
    \node [#1] at (path picture bounding box.center) {
      \includegraphics[#3]{#4}};
  }] {};}

\titleformat{\section}{
  \vspace{-6pt}\scshape\raggedright\large
}{}{0em}{}[\color{black}\titlerule \vspace{-5pt}]

\newcommand{\resumeItem}[2]{
\item\small{
    \textbf{#1}{: #2 \vspace{-2pt}}
  }
}

\newcommand{\resumeItemNoBullet}[2]{
\item[]\small{
    \hspace{-9pt}\textbf{#1}{: #2 \vspace{-6pt}}
  }
}

\newcommand{\resumeSubheading}[4]{
  \vspace{-1pt}\item[]
  \begin{tabular*}{0.98\textwidth}{l@{\extracolsep{\fill}}r}
    \hspace{-10pt}\textbf{#1} & #2 \\
    \hspace{-10pt}\textit{\small#3} & \textit{\small #4} \\
  \end{tabular*}\vspace{-5pt}
}

\newcommand{\resumeSubItem}[2]{\resumeItem{#1}{#2}\vspace{-4pt}}

\renewcommand{\labelitemii}{$\circ$}

\newcommand{\resumeSubHeadingListStart}{\begin{itemize}[leftmargin=*]}
  \newcommand{\resumeSubHeadingListEnd}{\end{itemize}}
\newcommand{\resumeItemListStart}{\begin{itemize}}
  \newcommand{\resumeItemListEnd}{\end{itemize}\vspace{-5pt}}

\newcommand{\shorterSection}[1]{\vspace{-10pt}\section{#1}}



\begin{document}

% ----------HEADING-----------------
\begin{figure}[!htb]
  \begin{minipage}{.5\textwidth}
    \begin{flushleft}
      \vspace{1.3cm}
      \small \textbf{\huge Stanislav Arnaudov} \\  \href{mailto:stanislav.arnaudov@kit.edu}{\color{blue}\underline{stanislav.arnaudov@kit.edu@kit.edu}} $\vert$
      LinkedIn: \href{https://www.linkedin.com/in/stanislav-arnaudov-37b475164/}{\color{blue}\underline{Arnaudov}} $\vert$\\
      Github: \href{https://github.com/palikar}{\color{blue}\underline{Arnaudov}} \\
      \small Däumlingweg 4,
      \small Karlsruhe 76199, Germany\\
    \end{flushleft}
  \end{minipage}%
  \begin{minipage}{0.5\textwidth}
    \begin{flushright}
      \roundpic[xshift=0cm,yshift=-0.5cm]{3.5cm}{width=0.37\linewidth, height=0.2\textheight}{fancy_stanche.jpg}
    \end{flushright}
  \end{minipage}
\end{figure}


% -----------EDUCATION-----------------
\shorterSection{Education}
\resumeSubHeadingListStart

\resumeSubheading
{Master of Science in Informatics}{Expected Sep 2021}
{Karlsruhe Institute of Technology}{Karlsruhe, Germany}
\resumeItemListStart
\resumeItem{Relevant Coursework}{Image processing, Computer Vision, Machine Learning, Software Engineering}
\resumeItem{Practical Courses}{Practice in Research}
\resumeItemListEnd

\vspace{10pt}

\resumeSubheading
{Bachelor of Technology in Informatics}{Sep 2015 - Sep 2018}
{Karlsruhe Institute of Technology}{Karlsruhe, Germany}
\resumeItemListStart
\resumeItem{Relevant Coursework}{Linear Algebra, Algorithms and Data Structures, Operating Systems, Software Engineering, Cognitive Systems, Computer Graphics, Mobile Computing, Databases}
\resumeItemListEnd


\resumeSubHeadingListEnd

% -----------SKILLS-----------------
\shorterSection{Skills}
\resumeSubHeadingListStart
\resumeSubheading{Programming Languages}{}{\vspace{-5pt}}{}
C++, Python, Java, JavaScript\textbackslash CSS\textbackslash HTML, Go, SQL, Emacs-Lisp
\resumeSubheading{Technologies}{}{\vspace{-5pt}}{}
Linux, Git, CMake, make, Robot Operating System (ROS), RabbitMQ, JavaFX/Java-Swing, JUnit, Maven, Frontend (AngularJS, VueJS), Backend (NodeJS, Express, Flask), OpenGL3/4, LaTeX, Emacs Org-mode, UML
\resumeSubheading{Software Libraries}{}{\vspace{-5pt}}{}
PyTorch, TensorFlow, Keras, Scikit-Learn, Numpy, Pandas, PyTorch, OpenCV, PCL (Point Cloud Library), OpenNI, GLFW, GLSL
\resumeSubHeadingListEnd
\vspace{7pt}

% -----------EXPERIENCE-----------------
\shorterSection{Experience}
\resumeSubHeadingListStart

\resumeSubheading
{Software Engineer\textbackslash Research Assistant}{Sep 2017 - Present}
{Fraunhofer IOSB}{Karlsruhe Germany}
\resumeItemListStart
\resumeItem{Image Processing}
{Working with OpenCV, implementing detection and tracking of a laser point.}
\resumeItem{Point Cloud Processing}
{Working with PCL, processing and using point-cloud information for automatic visual inspection systems.}
\resumeItem{Software Development}
{Developing and extending visual inspection systems for industrial applications.}
\resumeItemListEnd

\resumeSubheading
{Teaching Assistant in Linear Algebra}{Sep 2016 - Mar 2017}
{Karlsruhe Institute of Technology}{Karlsruhe Germany}
\resumeItemListStart
\resumeItem{Responsibilities}
{Checking homeworks and giving a class once a week.}
\resumeItemListEnd

\resumeSubheading
{Teaching Assistant in Algorithms and Data Structures}{Apr 2017 - Jul 2017}
{Karlsruhe Institute of Technology}{Karlsruhe Germany}
\resumeItemListStart
\resumeItem{Responsibilities}
{Checking homeworks and giving a class once a week.}
\resumeItemListEnd

\resumeSubheading
{Volunteer}{Jul 2018}
{Karlsruhe Institute of Technology}{Karlsruhe Germany}
\resumeItemListStart
\resumeItem{Responsibilities}
{Helping with the organization of the \href{https://cg.ivd.kit.edu/egsr18/}{\color{blue}\underline{EGSR 2018}} computer graphics conference.}
\resumeItemListEnd

\resumeSubHeadingListEnd


% -----------PROJECTS-----------------
\shorterSection{Projects}
\resumeSubHeadingListStart


\resumeSubItem{Practical Course in Scientific Research}
{\href{https://github.com/palikar/flow_predict}{\color{blue}\underline{Towards Bringing Together Numerical Methods for Partial Differential Equation and Deep Neural Networks}}
  \vspace{-5pt}
  \begin{itemize}
  \item Developing a personal research project.
  \item Investigating the current State-of-the-Art approaches.
  \item Developing a concrete research proposal.
  \item Conducting the actual project, performing experiments, summarizing results and drawing conclusions.
  \item Writing a paper that illustrates the results of the project.
  \end{itemize}
}

\resumeSubItem{Bachelor Thesis}
{
  \href{https://github.com/palikar/PollutionDev/}{\color{blue}\underline{Creating and Evaluating Stochastic Regression Models on the Basis of Heterogeneous Sensor Networks for Air Pollution}}
  \vspace{-5pt}
  \begin{itemize}
  \item Implementing stochastic regression models with Tensorflow, Edward and GPFlow.
  \item Evaluating stochastic regression models on the basis or proper scoring rules
  \item Writing out a thesis and presenting the collected results.
  \end{itemize}
}

\resumeSubItem{Practical Course in Software Engineering}
{NGram++
  \vspace{-5pt}
  \begin{itemize}
  \item Developing a single page application for analyzing and visualizing time series data.
  \item Designing and implementing the architecture of the application.
  \item Working in a team of 5 people.
  \end{itemize}
}

\resumeSubItem{Practical Course in Applied Geometry}
{C++ Geometry Library
  \vspace{-5pt}
  \begin{itemize}
  \item Modeling, analysis, reconstruction and simulation of geometric data.
  \item Extending a library by analyzing and implementing algorithms for B-splines, parallel curves, tensors surfaces and curvature visualization.
  \end{itemize}
}

\resumeSubItem{Course Project}
{\href{https://github.com/palikar/HomeworksSmart/}{\color{blue}\underline{Smart Homeworks}}
  \vspace{-5pt}
  \begin{itemize}
  \item Single page application for helping with the organization of homework assignments.\@
  \item Written in VueJS.\@
  \end{itemize}
}

\resumeSubItem{Personal Project}
{\href{https://github.com/palikar/alisp}{\color{blue}\underline{Alisp}}
  \vspace{-5pt}
  \begin{itemize}
  \item A general purpose programming language based on a Lisp
  \item Written in C++ and currently has around \numprint{25000} lines of code.\@
  \end{itemize}
}

\resumeSubItem{Personal Project}
{\href{https://github.com/palikar/anything}{\color{blue}\underline{Anything}}
  \vspace{-5pt}
  \begin{itemize}
  \item A 3d game engine written from scratch in C++ using OpenGL4.
  \end{itemize}
}

\resumeSubItem{Co-Maintainer of an Emacs package}
{\href{https://github.com/jaypei/emacs-neotree/}{\color{blue}\underline{Neotree}}
  \vspace{-5pt}
  \begin{itemize}
  \item Neotree - tree file browser for Emacs.
  \item Fixing bugs, implementing new features and helping with issues on the GitHub repository.
  \end{itemize}
}

\resumeSubHeadingListEnd

% -----------Addtional Experience & Achievements-----------------
\shorterSection{Additional Experience \& Achievements}

\resumeSubHeadingListStart
\item Co-author of a conference paper based on my bachelor thesis -- Stochastic Regression Models for Improving Data Quality, Calibration and Interpolation of Air Pollution Data from Distributed Sensor Networks of Low-Quality Sensors (\href{https://www.researchgate.net/publication/331131928_Stochastische_Regressionsmodelle_zur_Verbesserung_der_Datenqualitat_Kalibrierung_und_Interpolation_von_Umwelt-und_Luftdaten_in_verteilten_Messnetzen_aus_Low-Cost_Sensoren}{\color{blue}\underline{Researchgate Item}}).
\item Part of a team that ranked second in the (\href{https://www.meetup.com/Darmstadt-InnovationM-Round-Table/events/262603767/}{\color{blue}\underline{Code-2-Cloud Hackathon}}), organized by Merck and Accenture (8.07.2019 - 13.07.2019 in in Kronberg\textbackslash Darmstadt).
\item Doing Open Source as a hobby by fixing bugs and implementing features in different projects on {\href{https://github.com/palikar/}{\color{blue}\underline{GitHub}}.
\item Author of several small Emacs packages.
\item Spoken languages: German, English, Bulgarian
  \resumeSubHeadingListEnd
  
  % -------------------------------------------
\end{document}

%%% Local Variables:
%%% mode: latex
%%% TeX-master: t
%%% End:
